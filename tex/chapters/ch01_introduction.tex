\chapter{Introduction}

\section{Motivation}
Autonomous systems promise to improve our everyday life. They may support or replace human workers by working on dangerous, repetitive or physically challenging tasks. Real-life examples of such applications are robots that were used to clear debris at Fukushima nuclear site [TODO:ref], transporting mail in large logistics centers [TODO:ref] and transporting cargo over uneven terrain[TODO:ref].\\
In order to perform in real-life environments, such systems need to perceive their environment to a degree that allows them to identify hazards, obstacles and other objects that may affect their operations. Failure or inability to do so can result in temporary malfunction of robots, damage to the environment or workpieces or injury to humans. Thus, object recognition is key to employing autonomous systems.\\
Current research in the area of autonomous systems extends robotics by various approaches to artificial intelligence that are subject to research in machine learning; among them are neural networks that prove to perform well in object recognition [TODO:ref] but require a large amount of data that closely resembles reality, e.g. large sets of labelled photos, for training. In robotics though, coming by authentic data is not trivial as gathering data with robots requires real hardware and, depending on the tasks robots shall perform, can be time consuming.\\
Simulating robots in virtual environments, modeled to closely represent real environments, enables us to gather data without the need for any actual hardware. Virtual simulations that do not involve any non-virtual systems are not required to be run in real-time but can be sped up, resulting in potentially taking a fraction of the time needed for running real experiments.

\section{Purpose}
The purpose of this thesis is to develop a software environment that allows training and testing of image processing systems for use on autonomous robots.
% Constraints!
[TBD]

\section{Structure}
[TBD]