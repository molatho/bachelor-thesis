\chapter*{Abstract}
\subsubsection*{\hsmatitelen}
Everyday tasks performed by humans still prove to be a challenge to autonomous robots today due to the way they perceive their surroundings. Convolutional Neural Networks promise to enable machines to recognize well-known objects they were trained for. High quality datasets are required by this technology to produce reliable results. However these datasets are often hard to come by due to how much effort and time is needed for their production. While there are methods available to extend existing datasets, those are limited to primitive manipulations. The goal of this thesis is to conceptualize a process for automating the generation of high quality datasets by capturing images of virtual environments. These images shall feature sophisticated manipulations to individual objects, providing a large variety of input data for training Convolutional Neural Networks.

\subsubsection*{\hsmatitelde}
Während die Erfassung ihrer Umwelt für Menschen meist ein einfacher und zu weiten Teilen unterbewusster Prozess ist, stellt diese für Roboter bis heute eine große Herausforderung dar. Convolutional Neural Networks eröffnen Maschinen die Möglichkeit, bekannte und erlernte Objekte wiederzuerkennen. Um zuverlässige Ergebnisse liefern zu können, sind hochqualitative Datensätze erforderlich. Die Erstellung dieser ist jedoch sehr zeit- und arbeitsaufwändig. Zwar existieren bereits Methoden um bestehende Datensätze zu erweitern, jedoch beschränken sich diese auf zumeist primitive Manipulationen. Ziel dieser Arbeit ist die Konzeption zur automatischen Generierung hochqualitativer Bildaufnahmen unter der Verwendung von virtuellen Umgebungen. Einzelne Bilder unterscheiden sich hierbei durch anspruchsvolle Manipulationen individueller Objekte voneinander und tragen auf diese Weise zur Diversität so generierter Datensätze bei. Diese können für das Training von Convolutional Neural Networks verwendet werden.