% Die Arbeit besteht aus Kapiteln (chapter)
\chapter{Exemplaric Implementations}
Implementing the concept presented in \autoref{chap:conceptual-design} can be done in two parallel processes: writing the software itself and designing content such as 3D meshes of the robot and environment.

%////////////////////////////////////////////////
\section{Goal and constraints}
The goal of this implementation was to generate images for detection of doors in buildings. 

%////////////////////////////////////////////////
\section{Implementation I: Raycasting Using Blender}
Goal: Use Blender as an offline renderer, allowing for potentially photorealistic images being generated using a raycasting engine like cycles.

%////////////////////////////////////////////////
\section{Implementation II: Rasterization Using Unity}
Hypothesis: Rasterized images that are not photorealistic may still be of such high quality that they are sufficient for training CNNs. [TBD]
\subsection{Identifying Classifiers}
\subsubsection{Approach I: Using Coordinate-Transformations}
WorldToScreen (Bounding boxes/Colliders) [TBD]
\subsubsection{Approach II: Raycasting}
[TBD]

%////////////////////////////////////////////////

\subsection{Choice of Tools}
Comparision of engines (Unity, Unreal, Source, Cryengine, Frostbite) [TBD] \ref{table:game-engines}

\subsection{Designing content}
\subsubsection{Rebuilding a real environment}
The environment chosen for this implementation was the second floor of Building A of University of Applied Sciences Mannheim. [TBD]
\subsubsection{Optimizing Performance}
High-to-Lowpoly [TBD]