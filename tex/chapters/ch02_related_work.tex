\chapter{Related Work}

This chapter provides an overview about published work related to \acp{CNN} in the context of object detection and video games, \ac{ML} in the context of video games and robotics simulation. It aims to present the common past games and \ac{ML} share as well as modern approaches to combating today's problems using \ac{ML}.

\paragraph{Object detection Using Convolutional Neural Networks}
\acp{CNN} have their roots in the \emph{"NeoCognitron"} published by Fukushima in 1980 \cite{6313076}. In 1998, LeCun \textit{et al.} developed the network \textit{"LeNet5"} which focused on detection of handwritten digits \cite{Lecun98gradient-basedlearning}. Based on the architecture of LeNet5, Krizhevsky \textit{et al.} presented \emph{"AlexNet"} in 2012. The \emph{\ac{YOLO}} network by Redmond \textit{et al.} published in 2015 is an example of a modern \ac{CNN}: contrary to other detection systems that process images multiple times, \ac{YOLO} only processes an image once and hence runs considerably faster \cite{DBLP:journals/corr/RedmonDGF15}\cite{redmon2016yolo9000}\cite{yolov3}.
In 2017, Schweitzer wrote a master thesis about implementing "Real Time Object Detection for Autonomous Robots Using Neural networks on Mobile GPUs" that evaluated modern approaches to employ object detection using \acp{NN} \cite{Schweitzer2017}.

\paragraph{Cheating and Cheat Detection in Video Games}
Cheating is a common problem in multiplayer games today and has been in the past. Today, \acp{CNN} are used in cheat software for popular games such as "Counter-Strike: Global Offensive". By capturing the output of the game on the screen, \acp{CNN} are able to detect players in the local player's field of view and move the local player aim towards them \cite{Hayha}.\\
There are also attempts to employ \acp{NN} against cheating. According to McDonald, cheating in the computer game "Counter-Strike: Global Offensive" was the "largest self-reported problem by players in 2016" \cite{VACnet}. Through the implementation and usage of the neural network \emph{"VACnet"}, which observes players' behaviour and detects behaviour that is deemed artificial, McDonald concluded that "[with deeep learning] the battle against cheating is turning". Traditional approaches to combating cheating in games involved obtaining cheating software, manual reverse-engineering thereof and creating appropriate detection patterns to identify them running on machines. The neural network shifted the focus from analyzing software and handcrafting detection algorithms to analysis of players' behaviour.

\paragraph{Automating Play of Games}
The game \textit{"Super Mario Bros."} which ran on the \textit{"Nintendo Entertainment System"} and was released in 1985 made for an example of how the play of games could be automated. In 2009 Togelius \textit{et al.} evolved controllers based on \acp{NN} that controlled the inputs of a virtual console and in 2013, Murphy \textit{et al.} developed a general approach for playing \textit{Nintendo Entertainment System} games by deriving a function that deduces when a player is winning\cite{5286481}\cite{Tom13thefirst}. 

\paragraph{Robotics Simulation}
\emph{Robotics simulators} can be used to test the programming and bodies of virtual representations of robots in virtual environments. Their focus lies on simulating physics and kinematics in particular hence visual output is often basic and low-fidelity. In 2002, development on the high-fidelity robotics simulator \emph{"Gazebo"} began. Its focus lied on simulating outdoor environments \cite{Staranowicz2011}\cite{Gazebo}. Then in 2006 the Microsoft Corporation published the \emph{"Microsoft Robotics Developer Studio"} that featured a 3D simulator but introduced overhead in development such as a new programming language that was used to create applications and build user interfaces \cite{MRDS}. Its last stable version was released in 2012 and it has not been updated since. A notable robotics simulator is \emph{"OpenRAVE"} that was based on a dissertation by Diankov in 2010 and focuses on simulation of kinematic \cite{openRAVE}\cite{diankov_thesis}. \emph{"IKFast"} is a component of OpenRAVE that allows to analytically solve equations of kinematics chains and export code specific to individual equations. Today IKFast is also used in other software like the \ac{ROS} \cite{moveitikfast}.
The \emph{Robotics Toolbox} is a \emph{"Toolbox"} for the \emph{"MATLAB"} environment and was developed by Corke and first published in 1995, its latest stable version is version 10 which was released in 2017. It is a mathematical approach to robotics simulation that provides functions used for the study of robotics such as kinematics \cite{CorkePaper1995}\cite{CorkePaper1996}\cite{CorkematlabRoboticsToolboxBlog}.\\
In 2009, Seufert wrote a thesis about a "Conception of a Simulation Environment for Autonomous Mobile Robots" and in it concluded that by using a simulation environment, the productivity in developing and testing control software is increased as these environments do not require physical present robots \cite{Seufert2009}.