\chapter{Introduction}

\section{Motivation}
Autonomous systems promise to improve our everyday life. They may support or replace human workers by working on dangerous, repetitive or physically challenging tasks. Real-life examples of such applications are robots that were used to clear debris at Fukushima nuclear site \cite{fukushimaRobots}, transporting mail in large logistics centers \cite{dhlLogisticsRobots} and transporting cargo over uneven terrain \cite{RAIBERT200810822}.\\
In order to perform in real-life environments, such systems need to perceive their environment to a degree that allows them to identify hazards, obstacles and other objects that may affect their operations. Failure or inability to do so can result in temporary malfunction of robots, damage to the environment or workpieces or injury to humans \cite{robotVolkswagen}. Thus, object recognition is key to employing autonomous systems.\\
Current research in the area of autonomous systems extends robotics by various approaches to artificial intelligence that are subject to machine learning; among them are neural networks that prove to perform well in object recognition but require a large amount of data that closely resembles reality, e.g. large sets of labelled photos, for training. In robotics though, coming by authentic data is not trivial as gathering data with robots requires real hardware and, depending on the tasks robots shall perform, can be time consuming.\\
Simulating robots in virtual environments, modeled to closely represent real environments, enables us to gather data without the need for any actual hardware. Virtual simulations that only involve virtual systems are not required to be run in real-time but can be sped up, resulting in potentially taking a fraction of the time needed for running real experiments.

\section{Purpose and Structure of the Thesis}
The following chapter will show related work in the context of object detection using \acp{CNN}, video games and \ac{ML} and robotics simulators. Chapter \ref{chap:theoretical-background} provides insight into the underlying fundamentals this thesis is based on, including a brief take on the working principle of \ac{CNN}, transforms and operations performed on them, 3D graphics fundamentals and game engines. After that Chapter \ref{chap:conceptual-design} proposes a concept that describes a software environment, that allows to generate sets of labelled and diverse images, by defining high-level use-cases and deducing components from them. Subsequently, Chapter \ref{chap:implementations} demonstrates how the aforementioned concept can be implemented using game engines. Lastly, the thesis is summarized in Chapter \ref{chap:summary} and concluded in Chapter \ref{chap:conclusion}, providing insight into which goals were achieved by the implementation and how work on the concept and implementation can be continued.