\chapter{Conclusion}
\label{chap:conclusion}
% Abschließende Bewertung!
The goal of the thesis was to develop a concept for a software system that allows to generate diverse sets of labelled images of virtual environment using virtual robots that can be used for training \acp{CNN}. This concept defines abstract parts of a software architecture and thus is very flexible in the ways it can be implemented. Parts of it (such as the CaptureController) can even be implemented to run on multiple machines. Due to the concept being operating system, platform and framework agnostic, it can be implemented for a wide range of machines. Its high flexibility comes with one drawback though: demanding tasks that are not trivial to implement (such as identifying objects in images) need to be solved again for each implementation.\\
The implementation of this concept, \ac{VERE}, is operable in that it produces images, identifies objects in them and alters scenes stepwise, resulting in diverse datasets. In order to determine if the images are usable for training with \acp{CNN}, testing needs to be done. Also \ac{VERE}'s parameters (such as image resolution and scan resolution) need to be adjusted by testing various configurations and examining the output generated by actual \acp{CNN}: \acp{CNN} may not (yet) require high resolution images to produce satisfying results.\\
During the implementation of \ac{VERE} it became clear that this software required expert knowledge and skills from various sophisticated fields: a single person trying to become acquainted with and apply these proved to be difficult considering the time constraint. Reconstructing a detailed environment showed to require a considerable amount of effort and time: taking measure of the floor, windows, doors, door frames, desks and other objects was time-consuming but required for the implementation. The reconstruction was not limited to meshes but also included materials: photos needed to be taken of real surfaces (such as the floor and ceiling tiles), edited, imported into Unity and added to the individual objects in the scene. There are approaches that may greatly simplify these tasks:
\begin{description}
    \item [Procedurally Generated Geometry] Procedurally generating scenes by defining rules that specify how re-usable pieces of geometry and entities (\emph{prefabs}) are appended to form a scene is already used in games today \cite{strafeWiki}. Once these rules and prefabs are created, an arbitrary amount of scenes could be generated in a fraction of the time it would take designers to produce them. Future work on the concept presented in this thesis could evaluate if this approach is feasible and produces realistic results. Aspects that are of particular interest would be the complexity of the workflow of this approach and how it could be implemented so rules that were defined and prefabs that were created would be portable to multiple scenarios.
    \item [Using Pointcloud Data] To decrease the time spent on reconstructing real environments, pointcloud data of scans of existing environments could be used as blueprints. These would naturally contain relevant measures needed to reconstruct geometry, such as the dimensions of walls and floors. While this may sound trivial, gathering this data requires time and access to the environment and these may be limited resources. Reconstruction of environments by using pointcloud data is subject to current research and could be an interesting approach that future research might pick up and work on \cite{MINEO201981}. 
\end{description}
Implementing the concept should be done in a team of people that are experts at the tasks involved. Also by splitting tasks of the roles introduced by the concept (\textit{designers} and \textit{\ac{AI} engineers}) and sharing them with additional roles, work could be parallelized: \textit{3D artists} could reconstruct existing environments or model new ones while \textit{designers} specify, place and configure mutators required in a scene. \textit{Programmers} would implement and test those. \ac{AI} experts could use the time, that 3D artists, designers and programmers take to set up a scene, to develop a \ac{CNN} specific to the needs of the scenario.